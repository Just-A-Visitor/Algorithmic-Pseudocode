% Set the Page Layout
\documentclass[12pt]{article}
\usepackage[inner = 2.0cm, outer = 2.0cm, top = 2.0cm, bottom = 2.0cm]{geometry}

% Package to write pseudo-codes
\usepackage{algorithm}

% Remove the 'end' at the end of the algorithm
\usepackage{algpseudocode}

% Define Left Justified Comments
\algnewcommand{\LeftComment}[1]{\Statex \(\triangleright\) #1}

% Remove the Numbering of the Algorithm
\usepackage{caption}

\DeclareCaptionLabelFormat{algnonumber}{Algorithm}

\captionsetup[algorithm]{labelformat = algnonumber}

% Manually remove the 'end' for some sections
\algtext*{EndIf}
\algtext*{EndFor}
\algtext*{EndWhile}

% Define the command for a boldface 'is' 'to' 'downto'
\newcommand{\In}{\textbf{ in }}

\begin{document}

\begin{algorithm}

\caption{Find the maximum profit by buying and selling stocks}

\begin{algorithmic}[1]
    \Ensure One Based Indexing for the array
    \Statex
 
    \Function{$maxProfit$}{$arr$, $len$}
       
       \Statex
       \State $profit$ = 0
       
      \For{$i=1$ \textbf{in} $len-1$}	 
 	 	  \If{$arr$[$i$] $<$ $arr$[$i+1$]}
 	 	    \State $profit$ += $arr$[$i+1$] - $arr$[$i$]
 	      \EndIf
 	  \EndFor
      \State return $profit$

      \Statex
      \EndFunction
       
    \vskip 0.5in
    \subsection*{Walkthrough}
        \begin{enumerate}
            \item The parameter $arr$[$i$] represents the array of all the stock prices from day one to day $len$.
            \item The variable $profit$ is the profit, i.e. the sum of all \textbf{differences} of two bought and sold stocks.
            
            \item In order to maximize the profit, we iterate linearly through the stock $arr$ and always check, if the current index's stock is \textbf{cheaper}, than the succeeding index's stock. If so, this means that profit would be made, and the \textbf{difference} of the two prices is added to $profit$.
     \end{enumerate}


\end{algorithmic}
\end{algorithm}
\end{document}
