% Set the Page Layout
\documentclass[12pt]{article}
\usepackage[inner = 2.0cm, outer = 2.0cm, top = 2.0cm, bottom = 2.0cm]{geometry}


% Package to write pseudo-codes
\usepackage{algorithm}

% Remove the 'end' at the end of the algorithm
\usepackage[noend]{algpseudocode}

% Define Left Justified Comments
\algnewcommand{\LeftComment}[1]{\Statex \(\triangleright\) #1}

% Remove the Numbering of the Algorithm
\usepackage{caption}
\DeclareCaptionLabelFormat{algnonumber}{Algorithm}
\captionsetup[algorithm]{labelformat = algnonumber}

\begin{document}

\begin{algorithm}

  \caption{Calculate N-th term of the fibonacci sequence using a recursive dynamic programming approach.}
  \begin{algorithmic}[1]
    \Ensure Zero Based Indexing approach for the inicial vector (fib)
    \State $fib \gets [1, 1, 0, 0, 0, 0, \dots]$
    \Statex
    \Function{Fibonacci}{$n$}
        \If {$fib[n] <> 0$}
            \State \Return $fib[n]$
        \EndIf
        \State $fib[n] = Fibonacci[n - 2] + Fibonacci[n - 1]$
        \State \Return $fib[n]$
    \EndFunction
  \end{algorithmic}
  
\end{algorithm}

\end{document}
