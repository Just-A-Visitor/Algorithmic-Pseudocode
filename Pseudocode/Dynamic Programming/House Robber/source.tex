% Set the Page Layout
\documentclass[12pt]{article}
\usepackage[inner = 2.0cm, outer = 2.0cm, top = 2.0cm, bottom = 2.0cm]{geometry}


% Package to write pseudo-codes
\usepackage{algorithm}

% Remove the 'end' at the end of the algorithm
\usepackage[noend]{algpseudocode}

% Define Left Justified Comments
\algnewcommand{\LeftComment}[1]{\Statex \(\triangleright\) #1}

% Remove the Numbering of the Algorithm
\usepackage{caption}
\DeclareCaptionLabelFormat{algnonumber}{Algorithm}
\captionsetup[algorithm]{labelformat = algnonumber}

\begin{document}

\begin{algorithm}

  \caption{Find the maximum amount of money that can be robbed from non-adjacent houses}
\begin{algorithmic}[1]
    \Ensure One Based Indexing for the array
    \Statex
    
    \Function{$House\_Robber$}{$arr$}
    
        \LeftComment{$dp[i]$ represents the maximum money we can rob using the first $i$ houses}
        
        \Statex
        \State $n \gets arr.length$
        
        
        \State $dp[1] \gets max(0,arr[1])$
        \State $dp[2] \gets max(arr[1],arr[2])$
        
        
        \For {$i = 3$ \textbf{to} $n$}
            \State $take\_it \gets arr[i] + dp[i-2]$
            \State $leave\_it \gets dp[i-1]$
            \State $dp[i] \gets max(take\_it, \ leave\_it)$
        \EndFor
        
        \Statex
        
        \State \Return $dp[n]$
        
    \EndFunction
  \end{algorithmic}
  
\end{algorithm}


\end{document}
