% Set the Page Layout
\documentclass[12pt]{article}
\usepackage[inner = 2.0cm, outer = 2.0cm, top = 2.0cm, bottom = 2.0cm]{geometry}

% Package to write pseudo-codes
\usepackage{algorithm}

% Remove the 'end' at the end of the algorithm
\usepackage[noend]{algpseudocode}

% Define Left Justified Comments
\algnewcommand{\LeftComment}[1]{\Statex \(\triangleright\) #1}

% Remove the Numbering of the Algorithm
\usepackage{caption}
\DeclareCaptionLabelFormat{algnonumber}{Algorithm}
\captionsetup[algorithm]{labelformat = algnonumber}

\begin{document}

\begin{algorithm}
    
  \caption{Print the \textbf{Zig Zag} Traversal of a Binary Tree}
  
  
  \begin{algorithmic}[1]
    \Statex
    \Ensure The tree is \textbf{not} empty 
    
    \Statex
    \Function{Zig\_Zag\_Traversal}{$root$}
        \LeftComment $current\_stack$ contains all the elements of the current level. 
        \LeftComment $next\_stack$ contains all the elements of the next level.
        \LeftComment $left\_to\_right$ is true if the current level associates from left to right
        
        \Statex
        \State $current\_stack.push(root)$
        \State $left\_to\_right \gets True$
       
        \While{$current\_stack$ is \textbf{not empty, }}
            \While{$current\_stack$ is \textbf{not empty, }}
            
                \State $node \gets current\_stack.top$
                \State $current\_stack.pop$
                \State \textbf{Print}($node.data$)
                
                \If{$left\_to\_right$}
                    \If{Left Child Exists}
                        \State $next\_stack.push(node.left)$
                    \EndIf
                    \If{Right Child Exists}
                        \State $next\_stack.push(node.right)$
                    \EndIf
                \Else
                    \If{Right Child Exists}
                        \State $next\_stack.push(node.right)$
                    \EndIf
                    \If{Left Child Exists}
                        \State $next\_stack.push(node.left)$
                    \EndIf
                \EndIf
                
            \EndWhile
            \State \textbf{Print}(One level has been printed)
            \State $swap(current\_stack, next\_stack)$ \Comment{Go to the next level}
            \State $left\_to\_right \gets ! left\_to\_right$ \Comment{Change the associativity of the level}
        \EndWhile
    \EndFunction
  \end{algorithmic}
  
\end{algorithm}

\end{document}