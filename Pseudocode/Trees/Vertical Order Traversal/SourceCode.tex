% Set the Page Layout
\documentclass[12pt]{article}
\usepackage[inner = 2.0cm, outer = 2.0cm, top = 2.0cm, bottom = 2.0cm]{geometry}

% Package to write pseudo-codes
\usepackage{algorithm}

% Remove the 'end' at the end of the algorithm
\usepackage[noend]{algpseudocode}

% Define Left Justified Comments
\algnewcommand{\LeftComment}[1]{\Statex \(\triangleright\) #1}

% Remove the Numbering of the Algorithm
\usepackage{caption}
\DeclareCaptionLabelFormat{algnonumber}{Algorithm}
\captionsetup[algorithm]{labelformat = algnonumber}

\begin{document}

\begin{algorithm}

  \caption{Print the Vertical Order Traversal of a Tree}
  \begin{algorithmic}[1]
    \Require A non empty tree
    \Ensure In case of equal $x$ and $y$, the node which comes first
    in the level order traversal would be printed first
    \Statex
    \Function{Vertical\_Order\_Traversal}{$root$}
        \LeftComment \textit{Horiz\_Map} is a map with \textit{key} as the horizontal distance.  (can be negative)
        \LeftComment The \textit{value} is the vector of elements which are at the same horizontal level. 
        \LeftComment The elements in the vector are ordered according to their appearance in Level Order Traversal
        
        \Statex
        \State $horiz\_dist \gets 0$
        \State $queue.push(root,horiz\_dist)$
        
        \While{Queue is \textbf{not empty, }}
            \State $(current\_node, horiz\_dist) \gets queue.front $
            \State $queue.pop$
            \State $horiz\_map[horiz\_dist].push\_back(current\_node.data)$
            
            \If{Left Child Exists}
                \State $queue.push( current\_node.left, horiz\_dist-1 )$
            \EndIf
            
            \If{Right Child Exists}
                \State $queue.push(current\_node.right, horiz\_dist+1)$
            \EndIf
            
        \EndWhile
        \Statex
        \For{Sorted $Keys$ in $horiz\_map$}
            \For{all ordered elements in $horiz\_map[key]$ }
                \State \textbf{Print}($element)$ 
            \EndFor
        \EndFor
    \EndFunction
  \end{algorithmic}
  
\end{algorithm}

\end{document}
