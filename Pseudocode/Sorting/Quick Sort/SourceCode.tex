% Set the Page Layout
\documentclass[12pt]{article}
\usepackage[inner = 2.0cm, outer = 2.0cm, top = 2.0cm, bottom = 2.0cm]{geometry}

% Package to write pseudo-codes
\usepackage{algorithm}

% Don't Remove the 'end' at the end of the algorithm
\usepackage{algpseudocode}

% Manually remove the 'end' for some sections
\algtext*{EndIf}
\algtext*{EndFunction}
\algtext*{EndFor}

% Define Left Justified Comments
\algnewcommand{\LeftComment}[1]{\Statex \(\triangleright\) #1}

% Remove the Numbering of the Algorithm
\usepackage{caption}
\DeclareCaptionLabelFormat{algnonumber}{Algorithm}
\captionsetup[algorithm]{labelformat = algnonumber}

% Define the command for a boldface instructions
\newcommand{\Is}{\textbf{ is }}
\newcommand{\To}{\textbf{ to }}
\newcommand{\Downto}{\textbf{ downto }}
\newcommand{\Or}{\textbf{ or }}
\newcommand{\And}{\textbf{ and }}
% Use them inside Math-Mode (Hence the space!)

\begin{document}

\begin{algorithm}

  \caption{Sort an array using Divide and Conquer approach}
  
  \begin{algorithmic}[1]
    \Statex
    
    \Procedure{$Quick\_Sort$}{$arr, low, high$}
        \If{$low < high$}
            \State $pi = Partition(arr, low, high)$
            \State $Quick\_Sort(arr, low, pi - 1)$
            \State $Quick\_Sort(arr, pi + 1, high)$
        \EndIf
    \EndProcedure
    
    \Statex
    \Function{$Partition$}{$arr, low, high$}
        \State $pivot = arr[high]$
        \State $i = low - 1$
        \For {$j = low \To high - 1$}
            \If{$arr[j] < pivot$}
                \State $i++$
                \State $swap(arr[i],arr[j])$
            \EndIf
        \EndFor
        \State $swap(arr[i + 1],arr[high])$
        \State \Return $i + 1$
    \EndFunction
  \end{algorithmic}
  
\end{algorithm}

\end{document}
