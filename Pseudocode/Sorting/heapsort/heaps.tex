\documentclass[11pt]{article}

\usepackage{epsfig}
\usepackage{fullpage}

\begin{document}

\title{{\bf Heaps. Heapsort.}\\
\normalsize{(CLRS 6)}}
\date{}

\maketitle

%%%%%%%%%%%%%%%%%%%%%%%%%%%%%%%%%%%%%%%%%%%%%%%%%%%%%%%%%%%%%%%%%%%%%%%%%%

\section{Introduction}

\begin{itemize}
\item We have discussed several fundamental algorithms (e.g. sorting)
\item We will now turn to \emph{data structures}; Play an important role in
algorithms design.
        \begin{itemize}
        \item Today we discuss priority queues and next time
          structures for maintaining ordered sets.
        \end{itemize}
\end{itemize}

\section{Priority Queue}

\begin{itemize}
\item A priority queue supports the following operations on a set $S$ of
$n$ elements:
        \begin{itemize}
        \item {\sc Insert}: Insert a new element $e$ in $S$
        \item {\sc FindMin}: Return the minimal element in $S$
        \item {\sc DeleteMin}: Delete the minimal element in $S$
        \end{itemize}
\item Sometimes we are also interested in supporting the following
operations:
        \begin{itemize}
        \item {\sc Change}: Change the key (priority) of an element in $S$
        \item {\sc Delete}: Delete an element from $S$
        \end{itemize}
\item We can obviously sort using a priority queue:
        \begin{itemize}
        \item Insert all elements using {\sc Insert}
        \item Delete all elements in order using {\sc FindMin} and {\sc
        DeleteMin}
        \end{itemize}
\item Priority queues have many applications, e.g. in discrete event
simulation, graph algorithms
\end{itemize}


\subsection{Array or List implementations}
\begin{itemize}
\item The first implementation that comes to mind is ordered array: \\ \\
\centerline{\epsfig{file=Ordered_array.eps,height=0.5cm}}

        \begin{itemize}
        \item {\sc FindMin} can be performed in $O(1)$ time
        \item {\sc DeleteMin} and {\sc Insert} takes $O(n)$ time since we
        need to expand/compress the array after inserting or deleting element.
        \end{itemize}
\item If the array is unordered all operations take $O(n)$ time.
\item We could use double linked sorted list instead of array to avoid the
$O(n)$ expansion/compression cost
        \begin{itemize}
        \item but {\sc Insert} can still take $O(n)$ time.
        \end{itemize}
\end{itemize}


\subsection{Heap implementation}

\begin{itemize}
\item One way of implementing a priority queue is using a heap
\item Heap definition:

\fbox{\parbox{11.5cm}{
\begin{itemize}
\item Perfectly balanced binary tree
        \begin{itemize}
        \item lowest level can be incomplete (but filled from left-to-right)
        \end{itemize} 
\item For all nodes $v$ we have key($v$)$\geq$key(parent($v$))
\end{itemize}
}}

\item Example:

\vspace{.5\baselineskip}

\epsfig{file=heap,height=3cm}

\item Heap can be implemented (stored) in two ways (at least)
        \begin{itemize}
        \item Using pointers
        \item In an array level-by-level, left-to-right

        Example:

        \vspace{.5\baselineskip}

                \epsfig{file=arrayheap,height=3cm}

                \begin{itemize}
                \item the left and right children of node in entry $i$
                are in entry $2i$ and $2i+1$, respectively
	      \item the parent of node in entry $i$ is in entry
	      $\lfloor \frac{i}{2} \rfloor$
                \end{itemize}
        \end{itemize}
\item Properties of heap:
        \begin{itemize}
        \item Height $\Theta(\log n)$
        \item Minimum of $S$ is stored in root
        \end{itemize}
\item Operations:
        \begin{itemize}
        \item  {\sc Insert}
                \begin{itemize}
                \item Insert element in new leaf in leftmost possible
                position on lowest level
                \item Repeatedly swap element with element in parent node
                until heap order is reestablished ({\sc up-heapify})
		
                Example: Insertion of 4
		
                \epsfig{file=insert4.eps,height=3cm}
                \end{itemize}
        \item {\sc FindMin}
                \begin{itemize}
                \item Return root element
                \end{itemize}
                
        \item {\sc DeleteMin}
                \begin{itemize}
                \item Delete element in root
                \item Move element from rightmost leaf on lowest level to
                the root (and delete leaf)
                \item Repeatedly swap element with the smaller of the
                children elements until heap order is reestablished
                ({\sc down-heapify})

                \vspace{.5\baselineskip}

                Example: 

                \epsfig{file=deletemin.eps,height=3cm}
                \end{itemize}
        	
\item {\sc Change} and {\sc Delete} can be handled similarly in $O(\log n)$
time
\begin{itemize}
\item Note: Assuming that we know the element to be changed/deleted
(we cannot search in a heap!!)
\end{itemize}
\end{itemize}


\item {\bf Correctness:} Exercise.
\item {\bf Running time:} All operations traverse at most one
root-leaf path $\Rightarrow O(\log n)$ time.


\item Sorting using heap ({\em HeapSort}) takes $\Theta(n\log n)$ time.
  \begin{itemize}
  \item $n\cdot O(\log n)$ time to insert all elements (build the
    heap)
  \item $n\cdot O(\log n)$ time to output sorted elements
  \end{itemize}
  
\item Sometimes we would like to build a heap faster than $O(n\log n)$
\begin{itemize}
\item BUILDHEAP
  \begin{itemize}
  \item Insert elements in any order in perfectly balanced tree
  \item {\sc down-heapify} all nodes level-by-level, bottom-up
  \end{itemize}
  
\item Correctness: 
  \begin{itemize}
  \item Induction on height of tree: When doing level $i$, all trees
    rooted at level $i-1$ are heaps.
  \end{itemize}
  
\item Analysis:
  \begin{itemize}
  \item  The leaves are at height $0$, the root is at height $\log n$
  \item $n$ elements $\Rightarrow \leq \lceil \frac{n}{2}\rceil$
    leaves $\Rightarrow \lceil \frac{n}{2^h}\rceil$ elements at height 
    $h$
  \item Cost of {\sc down-heapify} on a node at height $h$ is $h$
  \item Total cost: $\sum_{i=1}^{\log n} h\cdot \lceil
    \frac{n}{2^h}\rceil=\Theta(n)\cdot \sum_{i=1}^{\log n}
    \frac{h}{2^h}$
  \item It can be shown that $\sum_{i=1}^{\log n} \frac{h}{2^h}=O(1)$
    $\Longrightarrow$ the total buildheap cost is $\Theta(n)$

    \vspace{\baselineskip}
  \item Computing $\sum_{i=1}^{n} \frac{h}{2^h}$ and
  $\sum_{i=1}^{\infty} \frac{h}{2^h}$
    \begin{itemize}
    \item Differentiate $\sum_{h=0}^{n} x^h = \frac{1 -
      x^{n+1}}{1-x}$, respectively $\sum_{h=0}^{\infty} x^h =
      \frac{1}{1-x}$ (assuming $|x|<1$)
    \item $\sum_{h=0}^{\infty} hx^{h-1}=\frac{1}{(x-1)^2} \Rightarrow
      \sum_{h=0}^{n} hx^h=\frac{x}{(x-1)^2} \Rightarrow \sum_{h=0}^{n}
      \frac{h}{2^h}=\frac{1/2}{(1/2-1)^2}=O(1)$
    \end{itemize}
  \end{itemize}
\end{itemize}

\end{itemize}
\end{document}
