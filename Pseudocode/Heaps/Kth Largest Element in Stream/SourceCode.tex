% Set the Page Layout
\documentclass[12pt]{article}
\usepackage[inner = 2.0cm, outer = 2.0cm, top = 2.0cm, bottom = 2.0cm]{geometry}

% Package to write pseudo-codes
\usepackage{algorithm}

% Remove the 'end' at the end of the algorithm
\usepackage[noend]{algpseudocode}

% Define Left Justified Comments
\algnewcommand{\LeftComment}[1]{\Statex \(\triangleright\) #1}

% Remove the Numbering of the Algorithm
\usepackage{caption}
\DeclareCaptionLabelFormat{algnonumber}{Algorithm}
\captionsetup[algorithm]{labelformat = algnonumber}

\begin{document}

\begin{algorithm}

  \caption{Kth Largest Element in a Stream}
  \begin{algorithmic}[1]
    \LeftComment $max\_heap$ buffers k elements from the stream.
    \LeftComment $max\_heap$ is a priority queue : The largest element is at the top
    \LeftComment $array$ must be at least $k-1$ elements long.
    \Statex
    \Function{Init\_Heap}{$array$, $k$}
        \State $max\_heap \gets$ Empty max\_heap
        \For{\textbf{each} element \textbf{in } $array$}
            \State $max\_heap.push(element)$
        \EndFor
    \EndFunction

    \Statex
    
    \Function{Add\_New\_Value\_From\_Stream}{$val$}
        \State $max\_heap.push(val)$
        \While{$max\_heap.size > k$}
            \State $max\_heap.pop$
        \EndWhile
        \State \Return{$max\_heap.top$}
    \EndFunction
  \end{algorithmic}
  
\end{algorithm}

\noindent
\textbf{Note} : The goal of the algorithm is to output the $Kth$ largest element as soon as we encounter a new element of the stream.
\end{document}
